\documentclass[a4paper, 12pt]{article}

\usepackage{color}
\usepackage[hidelinks]{hyperref}
\usepackage{url}
\usepackage[page]{appendix}
\usepackage[a4paper]{geometry}

\newcommand{\todo}[1]{\textbf{[[ TODO: }{\color{blue} #1}\textbf{]]}}
\newcommand{\fix}[1]{\textbf{[[ FIX: }{\color{red} #1}\textbf{]]}}

\newcommand{\eg}{\emph{e.g.}}
\newcommand{\ie}{\emph{i.e.}}

\newcommand{\jar}{\emph{jar}}
\newcommand{\ant}{Ant}
\newcommand{\maven}{Maven}
\newcommand{\gradle}{Gradle}
\newcommand{\sbt}{SBT}
\newcommand{\ivy}{Ivy}


\author{Jeanderson Barros Candido\\e-mail: \url{jeandersonbc@gmail.com}}

\title{Modernizing the Java Pathfinder Build Workflow: Migrating from \ant{} to
\gradle{}}

\date{\today}

\begin{document}

\maketitle

\section*{Summary}

Developers often performs recurrent tasks during the development process such
as testing, managing external libraries, generating API documentation, and
managing release artifacts.
Build tools help to automate those error-prone and daunt tasks.
Popular build tools usually provide a syntax to create a script file that
abstracts the commands to perform those tasks.
In the Java community, \ant{} used to be a popular choice of builder but many
projects have been replacing it in favor of other modern builders.
\emph{This proposal aims to modernize the build workflow from the Java
Pathfinder (JPF) project}.
To achieve this goal, I propose a smooth transitioning process to use
\gradle{}, a modern general purpose build tool.
This process consists in incrementally translating the existing \ant{} tasks
and comparing the result (\ie, differential testing) to the \ant{} outcome.
This proposal enumerates the tasks and sets the expectations to ensure
successful collaboration by the end of the Google Summer of Code 2018 edition.
% In the following, I summarize the structure of this proposal:
% Section~\ref{sec:motivation} highlights the issues with the current model and
% motivates the transition to a modern build tool, Section~\ref{sec:evaluation}
% evaluates how the alternatives performs in respect to the problems exposed on
% the previous section, and finally, Section~\ref{sec:plan} describes an
% execution plan (including tasks and schedule) and a checklist to measure my
% performance during the program.

\section{Introduction}
\label{sec:intro}

\ant{}\cite{page:ant} was released in 2000 and used to be a popular option to
automate build processes in the Java community.
However, with the release of more advanced builders, many Java projects have
been replacing Ant by other alternatives (\eg, \gradle{}\cite{page:gradle} or
\maven{}\cite{page:maven}).
This is understandable because \ant{} imposes some limitations that hinders the
developer's productivity in sufficiently complex/large projects compared to
those alternatives.
In the following, I elaborate two major issues in the context of the JPF
project.
Note that this is not an exhaustive and comprehensive list of disadvantages of
\ant{}.

\begin{enumerate}

\item \textbf{Lack of automatic dependency resolution.}
In the Java universe, there are several frameworks and libraries for different
purposes (\eg, testing and static analyzers) distributed as \emph{.jar} files.
Developers often rely on those libraries to reuse important functionalities
without reinventing the wheel.
For instance, JPF uses JUnit to implement and execute unit tests.
\ant{} requires the user to manually download and configure the desired
dependencies.
\ant{} is often integrated with \ivy{}\cite{page:ivy} as an complementary
tool to handle external dependencies.
On the other hand, \gradle{} and \maven{} (and other popular build tools)
resolve declared dependencies automatically out-of-the-box.

\item \textbf{Large and verbose script file.}
\ant{} uses an XML-based script file to define tasks and their settings (\eg,
\emph{classpath} and output directory).
XML is a widely disseminated format and requires little effort to understand
and use.
However, in the context of automating a build workflow, XML has some drawbacks.
XML tags are often long names.
In particular, \ant{} tasks may contain several attributes and nested
elements to describe additional properties.
For sufficiently large projects, it is challenging to maintain and evolve the
build process due to the quickly growth and the verbosity of the build script.

\item textbf{\fix{xxx}}
Although XML is a widely disseminated format that requires little effort to
understand and use, it often results in large and verbose files.
\fix{In addition, \ant{} lacks convention over configuration.
This means that one has to manually configure several properties (\eg,
\emph{classpath}) on the script file}

\end{enumerate}

Many popular build tools provide features to address those issues.
This project is relevant because the current JPF build workflow has error-prone
processes may introduce barriers to newcomers willing to contribute on the JPF
community.

There are several build tools available in the Java community.
\maven{} and \gradle{} are two mainstreams build tools popular in Android and
web development.
The most notable difference between them is the fact that \gradle{} uses a
Groovy-like DSL instead of XML.
\sbt{}\cite{page:sbt} is based on Scala and was suggested in the GSOC idea's
list\cite{page:jpf-gsoc18}.
Which one best fits JPF's needs?
To answer this question, \todo{what do I want to evaluate?}

%%  The \emph{usability} dimension evaluates how to implement basic functionalities
%%  (\eg, testing, compilation) and how to manage dependencies on each build tool.
%%  This is important to the maintenance and evolution of the script files.

%%  The \emph{popularity} dimension evaluates the engagement and maturity of the
%%  community.
%%  It is common to use Q\&A forums like StackOverflow\cite{page:stackoverflow} to
%%  find solutions to eventual problems and to learn how to perform some particular
%%  task ask questions and discover how to perform some task 

\section{Execution Plan}
\label{sec:plan}

\todo{wip}
\todo{it would be interesting to provide a checklist to mentors}

\bibliographystyle{plain}
\bibliography{references}

\end{document}
