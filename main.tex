\documentclass[a4paper]{article}

\usepackage{color}
\usepackage[hidelinks]{hyperref}
\usepackage{url}

\newcommand{\todo}[1]{\textbf{[[ TODO: }{\color{blue} #1}\textbf{]]}}
\newcommand{\fix}[1]{\textbf{[[ FIX: }{\color{red} #1}\textbf{]]}}
\newcommand{\eg}{\emph{e.g.}}
\newcommand{\ant}{Ant}
\newcommand{\maven}{Maven}
\newcommand{\gradle}{Gradle}

\author{Jeanderson Barros Candido\\e-mail: \url{jeandersonbc@gmail.com}}
\title{Modernizing Java Pathfinder Build Workflow}
\date{\today}

\begin{document}

\maketitle

\section*{Introduction}

Developers often performs recurrent tasks during the development process such
as testing, managing external libraries, generating API documentation, and
managing release artifacts.
Build tools help to automate those error-prone and daunt tasks.
Popular build tools usually provide a syntax to create configuration file
that abstracts the commands to perform those tasks.
In the Java community, \ant{} used to be a popular choice of builder but many
projects have been replacing it to other modern builders.
This proposal aims to modernize the build workflow from the Java Pathfinder
(JPF) project.
To achieve this goal, I propose a smooth transitioning process to use a modern
build tool and a list of tasks and expectations to ensure the success by the
end of the Google Summer of Code 2018.
In the following, I summarize the structure of this proposal:
Section~\ref{sec:motivation} highlights the issues with the current model and
motivates the transition to a modern build tool, Section~\ref{sec:evaluation}
evaluates how the alternatives performs in respect to the problems exposed on
the previous section, and finally, Section~\ref{sec:plan} describes an
execution plan (including tasks and scheduling) and a checklist to measure my
performance during the program.

\section{Motivation}
\label{sec:motivation}

\todo{Elaborate: Section~\ref{sec:motivation} highlights the issues with the
current model and motivates the transition to a modern build tool}

\section{Evaluation}
\label{sec:evaluation}

\todo{Elaborate: Section~\ref{sec:evaluation} evaluates how the alternatives
performs in respect to the problems exposed on the previous section}

\section{Execution Plan}
\label{sec:plan}

\todo{wip}
\todo{it would be interesting to provide a checklist for mentors}

\fix{Java Pathfinder (JPF) is an extensible model-checking framework for Java
programs, and it is capable to verify properties (\eg{}, null dereference) by
exploring multiple states of execution.
JPF became a relevant tool, widely explored in the academia, due to its adaptive
nature to different contexts.
The JPF community often reuse core functionalities and extend JPF to support
several types of analysis.}


\fix{
The goal of this project is to improve the JPF build system. Currently, JPF
uses Ant, and this project includes changing the JPF build system to sbt. This
also includes bringing the configuration mechanism of JPF under sbt. Currently,
the configuration mechanism is part of the core of JPF, jpf-core. The goal is
to make this functionally as part of the build system.
}

\end{document}
