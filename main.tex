\documentclass[a4paper, 12pt]{article}

\usepackage{color}
\usepackage[hidelinks]{hyperref}
\usepackage{url}
\usepackage[page]{appendix}
\usepackage[a4paper]{geometry}

\newcommand{\todo}[1]{\textbf{[[ TODO: }{\color{blue} #1}\textbf{]]}}
\newcommand{\fix}[1]{\textbf{[[ FIX: }{\color{red} #1}\textbf{]]}}

\newcommand{\eg}{\emph{e.g.}}
\newcommand{\ie}{\emph{i.e.}}

\newcommand{\jar}{\emph{jar}}
\newcommand{\ant}{Ant}
\newcommand{\maven}{Maven}
\newcommand{\gradle}{Gradle}
\newcommand{\sbt}{SBT}
\newcommand{\ivy}{Ivy}


\author{Jeanderson Barros Candido\\e-mail: \url{jeandersonbc@gmail.com}}
\title{Modernizing Java Pathfinder Build Workflow}
\date{\today}

\begin{document}

\maketitle

\section*{Introduction}

Developers often performs recurrent tasks during the development process such
as testing, managing external libraries, generating API documentation, and
managing release artifacts.
Build tools help to automate those error-prone and daunt tasks.
Popular build tools usually provide a syntax to create a script file that
abstracts the commands to perform those tasks.
In the Java community, \ant{} used to be a popular choice of builder but many
projects have been replacing it in favor of other modern builders.
\textbf{This proposal aims to modernize the build workflow from the Java
Pathfinder (JPF) project}.
To achieve this goal, I propose a smooth transitioning process to use a modern
build tool.
This process consists in incrementally translating the existing \ant{} tasks
and comparing the result (\ie, differential testing) to the \ant{} outcome.
This proposal enumerates the tasks and sets the expectations to ensure
successful collaboration by the end of the Google Summer of Code 2018 edition.
% In the following, I summarize the structure of this proposal:
% Section~\ref{sec:motivation} highlights the issues with the current model and
% motivates the transition to a modern build tool, Section~\ref{sec:evaluation}
% evaluates how the alternatives performs in respect to the problems exposed on
% the previous section, and finally, Section~\ref{sec:plan} describes an
% execution plan (including tasks and schedule) and a checklist to measure my
% performance during the program.

\section{Motivation}
\label{sec:motivation}

\ant{} was released in 2000 and used to be a popular option to automate
build processes in the Java community.
However, with the release of more advanced builders, many Java projects have
been replacing Ant by other alternatives (\eg, \gradle{} or \maven{}).
This is understandable because \ant{} imposes some limitations that hinders the
developer's productivity in sufficiently complex/large projects compared to
those alternatives.
In the following, I elaborate two major issues in the context of the JPF
project.
Note that this is not an exhaustive and comprehensive list of disadvantages of
\ant{}.

\begin{enumerate}

\item \textbf{Lack of automatic dependency resolution.}
In the Java universe, there are several frameworks and libraries for different
purposes (\eg, testing and static analyzers) distributed as \emph{.jar} files.
Developers often rely on those libraries to reuse important functionalities
without reinventing the wheel.
For instance, JPF uses JUnit to implement and execute unit tests.
\ant{} requires the user to manually download and configure the desired
dependencies.
\ant{} is often integrated with \ivy{} as an complementary tool to handle
external dependencies.
On the other hand, \gradle{} and \maven{} (and other popular build tools)
resolve declared dependencies automatically out-of-the-box.

\item \textbf{Verbose script file.}
\ant{} uses an XML-based script file to define tasks and their settings (\eg,
classpath and output directory).
Although XML is a widely disseminated format that requires little effort to
understand and use, it often results in large and verbose files.
In addition, \ant{} lacks convention over configuration.
This means that one has to manually configure several properties (\eg,
\emph{classpath}) on the script file.
Those additional configurations contribute to the size of the script file.

\end{enumerate}

Many popular build tools provide features to address those issues.
This project is relevant because the current JPF build workflow has error-prone
processes may introduce barriers to newcomers willing to contribute on the JPF
community.

\section{Evaluation}
\label{sec:evaluation}

There are several build tools available in the Java community (notably,
\maven{} and \gradle{}).
Which one best fits JPF's needs?
This section compares \maven{}, \gradle{}, and \sbt{}.


\todo{Elaborate: Section~\ref{sec:evaluation} evaluates how the alternatives
performs in respect to the problems exposed on the previous section}

\section{Execution Plan}
\label{sec:plan}

\todo{wip}
\todo{it would be interesting to provide a checklist for mentors}

\end{document}
