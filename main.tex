\documentclass{article}

\usepackage{color}
\usepackage[hidelinks]{hyperref}
\usepackage{url}
\usepackage[page]{appendix}
\usepackage[a4paper]{geometry}

\newcommand{\todo}[1]{\textbf{[[ TODO: }{\color{blue} #1}\textbf{]]}}
\newcommand{\fix}[1]{\textbf{[[ FIX: }{\color{red} #1}\textbf{]]}}

\newcommand{\eg}{\emph{e.g.}}
\newcommand{\ie}{\emph{i.e.}}

\newcommand{\jar}{\emph{jar}}
\newcommand{\ant}{Ant}
\newcommand{\maven}{Maven}
\newcommand{\gradle}{Gradle}
\newcommand{\sbt}{SBT}
\newcommand{\ivy}{Ivy}


\author{Jeanderson Barros Candido\\e-mail: \url{jeandersonbc@gmail.com}}
\title{Modernizing the Java PathFinder Build Workflow: Migrating from \ant{} to
\gradle{}}
\date{\today}

\begin{document}

\maketitle

\section*{Summary}

\noindent
Developers often performs recurrent tasks during the development process such
as testing, managing external libraries, generating API documentation, and
managing release artifacts.
Build tools help to automate those error-prone and daunt tasks.
Popular build tools usually provide a syntax to create a script file that
abstracts the commands to perform those tasks.
In the Java community, \ant{} used to be a popular choice but many
projects have been replacing it in favor of other modern builders.
\emph{This proposal aims to modernize the build workflow from the Java
PathFinder (JPF) project}.
To achieve this goal, I propose a smooth transitioning process to use
\gradle{}.
\gradle{} is a general purpose build system and uses Groovy, a JVM language, to
create and implement flexible and highly customizable build workflows.
This proposal enumerates the tasks and sets the expectations to ensure
successful collaboration by the end of the Google Summer of Code 2018 edition.

\section{Introduction}
\label{sec:intro}

\ant{}\cite{page:ant} was released in 2000 and used to be a popular option to
automate build processes in the Java community.
However, with the release of more advanced builders, many Java projects have
been replacing Ant by other alternatives (\eg, \gradle{}\cite{page:gradle} or
\maven{}\cite{page:maven}).
This is understandable because \ant{} has drawbacks that hinder developer's
productivity for sufficiently complex and large projects.
In the following, I elaborate two major issues in the context of the JPF
project.
Note that this is not an exhaustive and comprehensive list of disadvantages of
\ant{}.

\begin{enumerate}

\item \textbf{Lack of automatic dependency resolution.}
In the Java universe, there are several frameworks and libraries for different
purposes (\eg, testing and static analysis) distributed as \jar{} files.
Developers often rely on those libraries to reuse important functionalities
without reinventing the wheel.
For instance, JPF uses JUnit to implement and execute unit tests.
\ant{} requires the user to manually download and configure the desired
dependencies.
\ant{} is often integrated with \ivy{}\cite{page:ivy} as an complementary
tool to handle external dependencies.
On the other hand, \gradle{} and \maven{} (and other popular build tools)
resolve declared dependencies automatically out-of-the-box.

\item \textbf{Large and verbose script file.}
\ant{} uses XML to define tasks\footnote{To not confuse, this is not the same
as \ant{} tasks. In the context of \ant{}, build tasks are known as
\emph{targets} while commands (\eg, javac) are known as \emph{tasks}.} (\eg,
compile code) and their settings (\eg, \emph{classpath} and output directory).
XML is a widely disseminated format and requires little effort to understand
and use.
However, in the context of build automation, XML has some drawbacks.
XML tags are often long names.
In particular, \ant{} \emph{targets} may contain several attributes and nested
elements to describe additional properties.
For sufficiently large projects, it is challenging to maintain and evolve the
build process due to the quickly growth and the verbosity of the build script.

\end{enumerate}

Many popular build tools provide features to address those issues.
This proposal focuses on migrating from \ant{} to \gradle{}, and it is relevant
because the current JPF build workflow has error-prone processes that may
introduce barriers to newcomers willing to be part of the JPF community.
For further details on how and why I decided to migrate to \gradle{}, please,
refer to the Appendix~\ref{sec:eval}.

\section{Implementation Plan}
\label{sec:plan}

\fix{what am I going to do?}
\todo{check this link
\url{https://google.github.io/gsocguides/student/proposal-example-2}}

\section{Deliverables}
\label{sec:deliv}

\begin{enumerate}
\item ...
\end{enumerate}

\section{Timeline}
\label{sec:time}

\subsection*{Community Bonding}
\event{Apr/23 - Apr/27}{\todo{...}}
\event{Apr/30 - May/04}{\todo{...}}
\event{May/07 - May/13}{\todo{...}}

\subsection*{Coding}
\subsubsection*{Phase 1}
\event{May/14 - May/20}{\todo{...}}
\event{May/21 - May/27}{\todo{...}}
\event{May/28 - Jun/03}{\todo{...}}
\event{Jun/04 - Jun/10}{\todo{...}}

\event{Jun/11 - Jun/15}{First Evaluation}

\subsubsection*{Phase 2}
\event{Jun/11 - Jun/17}{\todo{...}}
\event{Jun/18 - Jun/24}{\todo{...}}
\event{Jun/25 - Jul/01}{\todo{...}}
\event{Jul/02 - Jul/08}{\todo{...}}
\event{Jul/09 - Jul/13}{Second Evaluation}

\subsubsection*{Phase 3}
\event{Jul/09 - Jul/15}{\todo{...}}
\event{Jul/16 - Jul/22}{\todo{...}}
\event{Jul/23 - Jul/29}{\todo{...}}
\event{Jul/30 - Aug/05}{\todo{...}}
\event{Aug/06 - Aug/14}{Code submission and final evaluation}

\clearpage

\appendix
\section{Appendix}
\subsection{Build Tools Evaluation}
\label{sec:eval}

There are several build tools available in the Java community.
\maven{} and \gradle{} are two mainstream tools popular in Android and web
development.
\sbt{}\cite{page:sbt} is another build tool that is becoming popular, and it
was suggested in the GSOC idea's list\cite{page:jpf-gsoc18}.
Which one fits better to JPF's needs?
To answer this question, I evaluated \maven{}, \gradle{}, and \sbt{} in respect
to the following aspects:

\begin{itemize}
\item \emph{Q1. How dependencies are managed?} Configuring paths and \jar{}
files manually can be error-prone. Ideally, the build tool would take care of
paths and dependency versions automatically.
\item \emph{Q2. How brief and powerful is the build script format?} XML syntax
may lead to overly verbose and large files for sufficiently large projects.
Ideally, the chosen build tool offers a friendly and brief syntax and provides
an easy way to create user-defined tasks.

\end{itemize}

\subsubsection*{Results}
\label{sec:results}

\begin{itemize}
\item \emph{Answering Q1:}
One of the features introduced by \maven{} was the automatic management of
external dependencies.
By default, \maven{} fetches \jar{} files from the Maven Central
Repository\cite{page:mvncentral} and keep them locally.
In addition, \maven{} allows access to other repositories\cite{page:mvnrepo}
with minor configuration.
Not surprisingly, \gradle{} and \sbt{} not only adopted this feature but
\emph{are also compatible} with \maven{} repositories.
Therefore, \textbf{all options are tied in this question}.

\item \emph{Answering Q2:}
In this aspect, \maven{} inherits drawbacks from \ant{} since it is also based
on XML.
The major difference of \gradle{} and \sbt{} is the fact that their scripts not
only \emph{describe} the build process but also allow to specify \emph{how}
tasks can be performed.
Build scripts are actual code.
In the case of \sbt{} and \gradle{}, both are compatible with Java API.
This feature opens many possibilities and conveniences to create custom build
processes and user-defined tasks.
Therefore, \textbf{\sbt{} (Scala-based) and \gradle{} (Groovy-based) are both
valid options}.

\end{itemize}

\subsubsection*{Conclusion}
\maven{} is a mature tool with many advantages compared to \ant{}.
However, the build script lacks \emph{expressiveness} since it is also based on
XML.
\sbt{}, on the other hand, empowers the developer by specifying the build
process with real code.
In addition, Scala features interoperability with Java.
\sbt{} is a promising tool but it is relatively recent.
The first stable release is from February 9\textsuperscript{th},
2018\cite{page:sbt-release}.
\textbf{\gradle{} is a mature tool and combines the best of \maven{} and \sbt{}}
including features like incremental building, a daemon for efficient
execution\cite{page:gradle-daemon}, and also a convenient \emph{wrapper}
mechanism\cite{page:gradle-wrapper} for reproducible builds.
\todo{mention Ant integration}
\textbf{For these reasons, \gradle{} seems to be a better fit for the context
of JPF.}
A list of useful links and snippets to demonstrate some \gradle{} basics are
available on GitHub\cite{page:gradle-labs} for further reference. 
Note that this evaluation is a sanity check and does not meant to be
generalized to all contexts in Java development.

\clearpage
\bibliographystyle{plain}
\bibliography{references}

\end{document}
